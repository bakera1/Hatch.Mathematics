\chapter{Introduction}
\label{ch:intro}

\section{Overview}

I wanted to write some brief notes to capture the key points covered in some of our weekly tutorials. Becoming a good mathematition is possible only by applying our recently acquired mathematical knowledge. For this reason these notes include some exercises to provide the reader a chance to build some skill in applying these new mathematical facts. 

In all cases getting mathematics wrong is a part of this learning process; we expect and need to fail to efficiently learn. The only thing that separates folks whom consider themselves capable in mathematics, is the level of persistence in the presence of this inevitable and expected failure. 

We should therefore always expect to get a little bit stuck; but try, try and try again!

\section{Fractions}

The value of a fraction is unaltered if the \textit{numerator} and \textit{denominator} are multiplied or divided by the same amount. It is important that we understand the basic operations of fractions, since not only are these operations used when we are performing simplifications of numeric only examples but these also constitute the same basic operations required when tackling symbolic or algebraic manipulation.

\subsection{Addition and Subtraction}

\begin{equation}
 \frac{x}{a} + \frac{y}{b} = \frac{x \times b + y \times a}{a \times b} = \frac{xb + ya}{ab}
\end{equation}

\begin{equation}
 \frac{x}{a} - \frac{y}{b} = \frac{x \times b - y \times a}{a \times b} = \frac{xb - ya}{ab}
\end{equation}

\subsection{Multiplication and Division}

\begin{equation}
 \frac{x}{a} \times \frac{y}{b} = \frac{x \times y}{a \times b} = \frac{xy}{ab}
\end{equation}

\begin{equation}
 \frac{x}{a} \div \frac{y}{b} = \frac{x}{a} \times \frac{b}{y} = \frac{x \times b}{a \times y} = \frac{xb}{ay}
\end{equation}

\subsection{Exercise fractions}

Simplify ...

\begin{equation}
  3\frac{1}{7} + \frac{2}{3}
\end{equation}

\begin{equation}
  3\frac{1}{7} - \frac{2}{3}
\end{equation}

\begin{equation}
  \frac{3}{7} \times \frac{1}{3}
\end{equation}

\begin{equation}
  \frac{3}{7} \div \frac{7}{3}
\end{equation}

\begin{equation}
  \frac{4x}{y} \times \frac{x}{6y}
\end{equation}

\begin{equation}
  2st \times \frac{3t}{s^{2}}
\end{equation}

\begin{equation}
  \frac{4 \pi r^{2}}{3} \div 2 \pi r
\end{equation}

\begin{equation}
  \frac{4uv}{3} \div \frac{u}{2v}
\end{equation}

\begin{equation}
  \frac{\pi x^{3}}{3} \div 8 \pi x
\end{equation}

\begin{equation}
  \frac{3x^{2}}{2y} \times \frac{y}{y-2}
\end{equation}

\section{Surds and Indices}

When we express a number as the product of two equal factors, that factor is called the \textit{square root} of the number, for example $ 4 = 2 \times 2 $ thus the square root of 4 is 2. This is written as $ 2 = \sqrt{4} $. Now -2 is also a square root of 4, as $ 4 = -2 \times -2 $. We can write that $ \pm \sqrt{4} = \pm 2 $. 

\subsection{Simplifying Surds}

Consider $ \sqrt{18} $ since one of the factors of 18 is 9 and 9 has an exact square root, 

\begin{equation}
 \sqrt{18} = \sqrt{9 \times 2} = \sqrt{9} \times \sqrt{2} 
\end{equation}

However since $ \sqrt{9} = 3 $ therefore $ 3 \times \sqrt{2} $ or $ 3\sqrt{2} $. Thus $ 3\sqrt{2} $ is the simplist possible form for the surd $ \sqrt{18} $. Similarly

\begin{equation}
 \sqrt{\frac{2}{25}} = \frac{\sqrt{2}}{\sqrt{25}} = \frac{\sqrt{2}}{5} 
\end{equation}

\subsubsection{Rationalising a Surd denominator}

In the example $ \frac{2}{\sqrt{3}} $ the square root in the denominator can be removed if we multiply it by another $ \sqrt{3} $.  Thus we can apply as below:

\begin{equation}
 \frac{2}{\sqrt{3}} = \frac{2 \times \sqrt{3}}{\sqrt{3} \times \sqrt{3}} = \frac{2\sqrt{3}}{3}
\end{equation}


\subsection{Exercise surds and indices}

Simplify ...

\begin{equation}
  \frac{\sqrt{80}}{\sqrt{16}}
\end{equation}

\begin{equation}
 \left ( \frac{\sqrt{80}}{3} + \frac{1}{3} \right)
\end{equation}

\begin{equation}
 \left ( \frac{\sqrt{80}}{3} + \frac{1}{6} \right)
\end{equation}

\begin{equation}
 \left (  \frac{\sqrt{80}}{3} + 2\frac{1}{6} \right )
\end{equation}

\begin{equation}
 \left (  \frac{\sqrt{100}}{20} + \frac{1}{8} + \frac{1}{8} \right )
\end{equation}

\begin{equation}
 \left (  \frac{\sqrt{150}}{8} + 7\frac{1}{16}\right )
\end{equation}

\begin{equation}
  \frac{1}{\sqrt{7}}
\end{equation}

\begin{equation}
  \frac{2}{\sqrt{11}}
\end{equation}

\begin{equation}
  \frac{3 \sqrt{2}}{\sqrt{5}}
\end{equation}

\begin{equation}
  \frac{\sqrt{5}}{\sqrt{10}}
\end{equation}

\begin{equation}
  \frac{\sqrt{1}}{\sqrt{27}}
\end{equation}

\section{Base and Index}

In an expression such as $ 3^{4} $ the \textit{base} is 3 and the 4 is called the \textit{power} or \textit{index}, working with indices involves using some properties which apply to any base, we we express these rules in terms of a general base $a$, which stands for any number.

\subsection{Rule 1}

Because $ a^{3} $ means $ a \times a \times a $ and $ a^{2} $ means $ a \times a $ it follows that  

\begin{equation}
  a^{3} \times a^{2} = ( a \times a \times a ) \times ( a \times a ) = a^{5}
\end{equation}

More generally ...

\begin{equation}
  a^{p} \times a^{q} = a^{p + q}
\end{equation}

\subsection{Rule 2}

Dealing with division

\begin{equation}
  a^{7} \div a^{4} =\frac{\cancel{a} \times \cancel{a} \times \cancel{a}\times \cancel{a}\times a\times a\times a}{\cancel{a}\times \cancel{a}\times \cancel{a}\times \cancel{a}} = a^{3}
\end{equation}

this can be also be read as $ a^{7} \div a^{4} = a^{7-4} $ or more generally ...

\begin{equation}
  a^{p} \div a^{q} = a^{p - q}
\end{equation}

Thus from rule 2 say:

\begin{equation}
  a^{3} \div a^{5} = a^{3 - 5} = a^{-2} = \frac{1}{a^{2}}
\end{equation}

This tells us that $a^{-2}$ means $ \frac{1}{a^{2}} $ which more generally 

\begin{equation}
  a^{-p} = \frac{1}{a^{p}}
\end{equation}

This $a^{-p}$ means the reciprocal of $a^{p}$. 

Finally for rule 2 any base to the power zero is equal to 1. In fact any number at all raised to the power zero is always 1. 

\begin{equation}
  a^{0} = 1
\end{equation}

\subsection{Rule 3}

\begin{equation}
 (a^{2})^{3}  = (a \times a)^{3}
\end{equation}

\begin{equation}
(a \times a)^{3} = (a \times a) \times (a \times a) \times (a \times a)= a^{6}
\end{equation}

More generally ...

\begin{equation}
(a^{p})^{q} = a^{p \times q}
\end{equation}

Not that this is different to rule 2, since in rule 3 case we have $ (a^{p})^{q} $ where $ (a^{p})$ is raised to the power of $q$ relative to previous rule 2 case where $ a^{p} \times a^{q} $.

\subsection{Rule 4}

This rule explains the meaning of a fractional index

\begin{equation}
a^{\frac{1}{2}} \times a^{\frac{1}{2}} = a^{\frac{1}{2} + \frac{1}{2}} = a^{1} = a 
\end{equation}

Thus

\begin{equation}
a = a^{\frac{1}{2}} \times a^{\frac{1}{2}}  
\end{equation}

Therefore $a^{\frac{1}{2}}$ means $\sqrt{a} $ the positive square root of a 

\begin{equation}
a = a^{\frac{1}{2}} \times a^{\frac{1}{2}}  = \sqrt{a} \times \sqrt{a} = a^{1} = a
\end{equation}

More generally ...

\begin{equation}
a^{\frac{p}{q}} = (a^{p})^{\frac{1}{q}} 
\end{equation}

or

\begin{equation}
a^{\frac{p}{q}} = (a^{\frac{1}{q}})^{p}
\end{equation}

\subsection{Exercise base and index}

Simplify ...

\begin{equation}
  \frac{2^{3} \times 2^{7}}{4^{3}}
\end{equation}

\begin{equation}
  (x^{2})^{7} \times x^{-3}
\end{equation}

\begin{equation}
  (x^{2})^{7} \times x^{-3}
\end{equation}

